%#!lualatex
% Time-Stamp: <2022-11-23 15:18:20>
\documentclass[11pt,pdfa,ja]{MishoText}
\usepackage{lipsum}
\title{Plain JA-Sample for MishoText}
\author{Sho Iwamoto}

\hypersetup{
  pdflang={ja},
  pdfauthortitle={Postdoctoral Researcher, ELTE Eötvös Loránd University},
  pdfcopyright={2022 Sho Iwamoto},
  pdfcontactemail={sho.iwamoto@ttk.elte.hu},
  pdfcontacturl={https://www.misho-web.com/},
  pdflicenseurl={https://www.misho-web.com/},
}

\setotherlanguage{hebrew}

\newfontfamily{\hebrewfont}{AdobeHebrew-Regular.otf}
\newfontfamily{\hebrewfontsf}{AdobeHebrew-Regular.otf}
\newfontfamily{\hebrewfonttt}{AdobeHebrew-Regular.otf}

\begin{document}
\maketitle

\chapter{日本語とか英語とか}
\section{日本語の文章}
親譲りの\textbf{無鉄砲}で小供の時から損ばかりしている。小学校に居る時分学校の二階から飛び降りて一週間ほど腰を抜かした事がある。なぜそんな無闇をしたと聞く人があるかも知れぬ。別段深い理由でもない。新築の二階から首を出していたら、同級生の一人が冗談に、いくら威張っても、そこから飛び降りる事は出来まい。弱虫やーい。と囃したからである。小使に負ぶさって帰って来た時、おやじが大きな眼をして二階ぐらいから飛び降りて腰を抜かす奴があるかと云ったから、この次は抜かさずに飛んで見せますと答えた。

髙橋さん親類のものから西洋製のナイフを貰って奇麗な刃を日に翳して、友達に見せていたら、一人が光る事は光るが切れそうもないと云った。切れぬ事があるか、何でも切ってみせると受け合った。そんなら君の指を切ってみろと注文したから、何だ指ぐらいこの通りだと右の手の親指の甲をはすに切り込んだ。幸ナイフが小さいのと、親指の骨が堅かったので、今だに親指は手に付いている。しかし創痕は死ぬまで消えぬ。

庭を東へ二十歩に行き尽すと、南上がりにいささかばかりの菜園があって、真中に栗の木が一本立っている。これは命より大事な栗だ。実の熟する時分は起き抜けに背戸を出て落ちた奴を拾ってきて、学校で食う。菜園の西側が山城屋という質屋の庭続きで、この質屋に勘太郎という十三四の倅が居た。勘太郎は無論弱虫である。弱虫の癖に四つ目垣を乗りこえて、栗を盗みにくる。ある日の夕方折戸の蔭に隠れて、とうとう勘太郎を捕まえてやった。その時勘太郎は逃げ路を失って、一生懸命に飛びかかってきた。向うは二つばかり年上である。弱虫だが力は強い。鉢の開いた頭を、こっちの胸へ宛ててぐいぐい押した拍子に、勘太郎の頭がすべって、おれの袷の袖の中にはいった。邪魔になって手が使えぬから、無暗に手を振ったら、袖の中にある勘太郎の頭が、右左へぐらぐら靡いた。しまいに苦しがって袖の中から、おれの二の腕へ食い付いた。痛かったから勘太郎を垣根へ押しつけておいて、足搦をかけて向うへ倒してやった。山城屋の地面は菜園より六尺がた低い。勘太郎は四つ目垣を半分崩して、自分の領分へ真逆様に落ちて、ぐうと云った。勘太郎が落ちるときに、おれの袷の片袖がもげて、急に手が自由になった。その晩母が山城屋に詫びに行ったついでに袷の片袖も取り返して来た。

\begin{english}
\section{English Text}
It is a \textbf{long established} fact that a reader will be distracted by the readable content of a page when looking at its layout. The point of using Lorem Ipsum is that it has a more-or-less normal distribution of letters, as opposed to using 'Content here, content here', making it look like readable English. Many desktop publishing packages and web page editors now use Lorem Ipsum as their default model text, and a search for 'lorem ipsum' will uncover many web sites still in their infancy. Various versions have evolved over the years, sometimes by accident, sometimes on purpose (injected humour and the like).

There are many variations of passages of Lorem Ipsum available, but the majority have suffered alteration in some form, by injected humour, or randomised words which don't look even slightly believable. If you are going to use a passage of Lorem Ipsum, you need to be sure there isn't anything embarrassing hidden in the middle of text. All the Lorem Ipsum generators on the Internet tend to repeat predefined chunks as necessary, making this the first true generator on the Internet. It uses a dictionary of over 200 Latin words, combined with a handful of model sentence structures, to generate Lorem Ipsum which looks reasonable. The generated Lorem Ipsum is therefore always free from repetition, injected humour, or non-characteristic words etc.
\end{english}

\chapter{Hebrew and Latin}
\begin{hebrew}
\section{עברית}
הכתבים המוקדמים ביותר המשתמשים באלפבית העברי נכתבו בכתב העברי הקדום ששימש עד לתקופת בית שני. הכתב העברי הקדום הוא נוסח מקומי של האלפבית הפיניקי, שבתורו התפתח מהאלפבית הפרוטו-כנעני שפותח בהשראת ההירוגליפים המצרים.

בתקופת בית שני אומץ האלפבית הארמי לשימוש השפה העברית במקום האלפבית העברי העתיק, וממנו התפתח הכתב המרובע. בתקופת המעבר נעשה שימוש מועט בכתב העברי הקדום, למשל בכתיבת השמות הקדושים והטבעת מטבעות. עם הזמן, נעלם גם שימוש זה של הכתב העברי הקדום. האלפבית העברי של ימינו הוא פיתוח של הכתב המרובע. היו שטענו שהמעבר לשימוש בכתב המרובע נבע מתוך רצון להיבדל מהשומרונים שהחזיקו בצורה ייחודית של הכתב העברי העתיק[דרוש מקור].

במקביל לכתב המרובע, המשמש בעיקר לדפוס, התפתח כתב רהוט, המתאפיין בקווים מעוגלים, לצורך כתיבה מהירה, והוא נפוץ מאוד בכתבי יד לא-מודפסים. וואריאציות אחרות של הכתב העברי הם כתב ספרדי, כתב איטלקי וכתב חצי קולמוס.

במהלך הדורות השתכלל הכתב העברי המרובע וזכה לגופנים רבים ומגוונים. בין החשובים והידועים שבהם הוא כתב רש"י, שמקורו בסגנון כתיבה מספרד שהיה נהוג בימי הביניים. 
\end{hebrew}

\lipsum[1]

\end{document}
