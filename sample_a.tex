%#!lualatex
\documentclass[pdfa]{MishoText}
\usepackage{lipsum}
\setotherlanguages{japanese}
\title{Sample for MishoText}
\author{Sho Iwamoto}

\hypersetup{
  pdflang={en−US},
  pdfkeywords={Manual},
  pdfsubject={LaTeX},
  pdfcopyright={2022 Sho Iwamoto},
  pdfcontactemail={sho.iwamoto@ttk.elte.hu},
  pdfcontacturl={https://www.misho-web.com/},
  pdflicenseurl={https://www.misho-web.com/},
  pdfpubstatus={AO},
}


\begin{document}
\maketitle

\chapter{Basic information}
\label{cha:basic-information}

This is a sample file.

\begin{itemize}
 \item Baselineskip: \ShowLength{\baselineskip}
 \item TextWidth: \ShowLength{\textwidth} among 210mm
 \item TextHeight: \ShowLength{\textheight} among 297mm
\end{itemize}

\makeatletter\ShowLength{\@total@headsep}\makeatother

Lorem ipsum dolor sit amet, consectetuer adipiscing elit. Ut purus elit, vestibulum ut, placerat ac, adipiscing vitae, felis. Curabitur dictum gravida mauris.
And Hyperlink:~\ref{cha:basic-information} and \url{https://www.misho-web.com/} \cite{hogehoge}.

Color samples: \BLUE{abc}\PINK{abc}\GREEN{abc}\C{$\int x$ and y}\C*{$\gamma$ not true}

(orange!30) \colorbox{pLightOrange!30}{abc\BLUE{def}\PINK{ghijk}\GREEN{lmnop}\C{$\int x$ and y}\C*{$\gamma$ not true}}

(purple!30) \colorbox{pLightPurple!30}{abc\BLUE{def}\PINK{ghijk}\GREEN{lmnop}\C{$\int x$ and y}\C*{$\gamma$ not true}}

(green!30) \colorbox{pLightGreen!30}{abc\BLUE{def}\PINK{ghijk}\GREEN{lmnop}\C{$\int x$ and y}\C*{$\gamma$ not true}}

\chapter{Fonts and Symbols}

\section{Font families for Latin}

\def\w#1{#1\textsf{#1}\texttt{#1} }
\w a\w b\w c\w d\w e\w f\w g\w h\w i\w j\w k\w l\w m\w n\w o\w p\w q\w r\w s\w t\w u\w v\w w\w x\w y\w z\w 1\w 2\w 3\w 4\w 5\w 6\w 7\w 8\w 9\w0 \w+\w-

\section{Greek letters}
\subsection{Unicode text}
\begin{itemize}
  \item normal: αβγδϵζηθικλμνξoπρστυϕχψω
  \item italic: \textit{αβγδϵζηθικλμνξoπρστυϕχψω}
  \item bold: \textbf{αβγδϵζηθικλμνξoπρστυϕχψω}
  \item bold italic: \textbf{\textit{αβγδϵζηθικλμνξoπρστυϕχψω}}
\end{itemize}

\subsection{Big math and Math symbols}
\[
  \text{normal: }\Gamma \Delta \Theta \Lambda \Xi \Pi \Sigma \Upsilon \Phi \Chi \Psi \Omega
  \alpha \beta \gamma \delta \epsilon \zeta \eta \theta \iota \kappa \lambda \mu \nu \xi \pi \rho \sigma \tau \upsilon \phi \chi \psi \omega
  \varphi\varepsilon\varsigma
\]
\[\mathbf{
  \text{bf: }
  \Gamma \Delta \Theta \Lambda \Xi \Pi \Sigma \Upsilon \Phi \Chi \Psi \Omega
  \alpha \beta \gamma \delta \epsilon \zeta \eta \theta \iota \kappa \lambda \mu \nu \xi \pi \rho \sigma \tau \upsilon \phi \chi \psi \omega
  \varphi\varepsilon\varsigma
}\]
\[\mathup{
  \text{upright: }
  \alpha \beta \gamma \delta \epsilon \zeta \eta \theta \iota \kappa \lambda \mu \nu \xi \pi \rho \sigma \tau \upsilon \phi \chi \psi \omega
  \varphi\varepsilon\varsigma
}\]
\[\mathbf{\mathbfup{
  \text{bfup: }
  \Gamma \Delta \Theta \Lambda \Xi \Pi \Sigma \Upsilon \Phi \Chi \Psi \Omega
  \alpha \beta \gamma \delta \epsilon \zeta \eta \theta \iota \kappa \lambda \mu \nu \xi \pi \rho \sigma \tau \upsilon \phi \chi \psi \omega
  \varphi\varepsilon\varsigma
}}\]

\section{Math accents}
\begin{equation}
 \sqrt{\sqrt{\sqrt{\sqrt{\sqrt{\sqrt{\sqrt{\sqrt{\frac ab}}}}}}}} \bar v=\iiint\overline{v}
\end{equation}
\begin{equation}
 \hat{\bar{x}} + \overline{\acute{y}} = ☉♁☾⛢♂
\end{equation}

\section{Notes}
\lipsum[1]

\Note{hogehogei $h^{h^h}$ \lipsum[1]}

\Remark{hogehoge!}

\begin{DownPara}
\lipsum[2]

\lipsum[1]
\end{DownPara}

\lipsum[1]

Automatic note after this paragraph\addnote{Automatic note after this paragraph}.
Multiple notes\addnote{ more than two, as well} is possible.

\OutputNote

Automatic note after this paragraph\addnote{Automatic note after this paragraph}.
Multiple notes\addnote{more than two, as well} is possible.

\OutputNote
\OutputNote
\OutputNote
\OutputNote

\begin{align}
1+1=2\addnote*{even in eqs, but only once}
\end{align}
\OutputNote


\begin{DownPara}
\paragraph{Lipsum}\lipsum[2]

\lipsum[1]
\end{DownPara}

\paragraph{Lipsum}
\lipsum\lipsum\lipsum


\chapter{Minted}

\lipsum[1]

\texttt{This is a code with typewriter font.}
\begin{minted}{wolfram}
ClearProcess[];
topologies = CreateTopologies[1, 1->2, ExcludeTopologies->{Internal}];
diagrams = InsertFields[topologies, {F[2,{2}]}->{F[2,{2}], V[1]}, ExcludeParticles->{S, V[2|3|4|10]}];
oneQED = CalcFeynAmp[OffShell[CreateFeynAmp[diagrams],3->q]];
!$R_{\epsilon''}$! = RotationMatrix[!$\epsilon''$!, {0, 1, 0}];
!$R_{\theta}$!  = RotationMatrix[!$\theta$!, {0, 0, 1}];
!$R_{\delta}$!  = RotationMatrix[!$\delta$!, {1, 0, 0}];
\end{minted}
\begin{DownPara}
 The code should be aligned with \code|DownPara| env.
\end{DownPara}
\Note{And \code|Note| env.}



\begin{thebibliography}{9}
\bibitem{hogehoge}
Donald E. Knuth (1986) \emph{The \TeX{} Book}, Addison-Wesley Professional.
\end{thebibliography}

\end{document}

