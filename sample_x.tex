%#!lualatex
\documentclass[11pt,pdfx]{MishoText}
\usepackage{lipsum}
\setotherlanguages{japanese,latin}
\title{Sample for MishoText}
\author{Sho Iwamoto}

\hypersetup{
  pdflang={en−US},
  pdfkeywords={Manual},
  pdfsubject={LaTeX},
  pdfcopyright={2022 Sho Iwamoto},
  pdfcontactemail={sho.iwamoto@ttk.elte.hu},
  pdfcontacturl={https://www.misho-web.com/},
  pdflicenseurl={https://www.misho-web.com/},
  pdfpubstatus={AO},
}

\newcommand\CMD[1]{\texttt{\textbackslash#1}}
\newcommand\PKG[1]{\texttt{#1}}
\begin{document}
\maketitle

\chapter{Basic information}
\label{cha:basic-information}

This is a sample file. The dimension is given by
\begin{itemize}
 \item Baselineskip: \ShowLength{\baselineskip}
 \item TextWidth: \ShowLength{\textwidth} among 210mm
 \item TextHeight: \ShowLength{\textheight} among 297mm
\end{itemize}
We can use \CMD{ShowLength} to display the length. For example, \CMD{@total@headsep} is displayed by:
\makeatletter\ShowLength{\@total@headsep}\makeatother

\section{Hyperref}\label{sec:beginning}
\PKG{hyperref} package is used to provide links: Chapter~\ref{cha:basic-information}, Section~\ref{sec:beginning}, URLs such as \url{https://www.misho-web.com/} or \href{https://www.google.com/}{Google}, and for citations \cite{hogehoge}.

\section{Colors}
\begin{tabbing}
Color samples:\qquad\quad\= abc\BLUE{def}\PINK{ghijk}\GREEN{lmnop}\C{$\int x$ and y}\\
(\texttt{pLightOrange!30}) \> \colorbox{pLightOrange!30}{abc\BLUE{def}\PINK{ghijk}\GREEN{lmnop}\C{$\int x$ and y}}\\
(\texttt{pLightPurple!30}) \> \colorbox{pLightPurple!30}{abc\BLUE{def}\PINK{ghijk}\GREEN{lmnop}\C{$\int x$ and y}}\\
(\texttt{pLightGreen!30})  \> \colorbox{pLightGreen!30}{abc\BLUE{def}\PINK{ghijk}\GREEN{lmnop}\C{$\int x$ and y}}\\
\end{tabbing}
The command \CMD{C} is used to give alternative definitions with {\color{AltDefA}AltDefA} (=\CMD{red}\footnote{\texttt{\#ff3300} instead of pure red.}). It penetrates into math expressions:
\begin{align*}
\C x && \C x^2 && \C ax^2 && \C f_x^2 && \C f^2_x && \C a'_x && \C a_x' && \C a''_x
\end{align*}
Historically, \CMD{C*} is prepared with {\color{AltDefB}AltDefB} color but it is unused and hidden.
\chapter{Fonts and Symbols}

\section{Font families}
\subsection{Latin: serif, sans-serif, and typewriter}

\def\w#1{#1\textsf{#1}\texttt{#1} }
\w a\w b\w c\w d\w e\w f\w g\w h\w i\w j\w k\w l\w m\w n\w o\w p\w q\w r\w s\w t\w u\w v\w w\w x\w y\w z\w 1\w 2\w 3\w 4\w 5\w 6\w 7\w 8\w 9\w0 \w+\w-

\subsection{Greek: Unicode-text version (and Chinese/Japanese)}
\begin{itemize}
  \item normal: αβγδϵζηθικλμνξoπρστυϕχψω
  \item italic: \textit{αβγδϵζηθικλμνξoπρστυϕχψω}
  \item bold: \textbf{αβγδϵζηθικλμνξoπρστυϕχψω}
  \item bold italic: \textbf{\textit{αβγδϵζηθικλμνξoπρστυϕχψω}}
  \item Chinese letters: \japanese{東東东, 万萬万, 愛愛爱, 楽樂乐, 嬢孃娘, 豊豐丰, 単單单}\endjapanese
\end{itemize}

\subsection{Greek: Math-symbol version}
\[
  \text{normal: }\Gamma \Delta \Theta \Lambda \Xi \Pi \Sigma \Upsilon \Phi \Chi \Psi \Omega
  \alpha \beta \gamma \delta \epsilon \zeta \eta \theta \iota \kappa \lambda \mu \nu \xi \pi \rho \sigma \tau \upsilon \phi \chi \psi \omega
  \varphi\varepsilon\varsigma
\]
\[\mathbf{
  \text{bf: }
  \Gamma \Delta \Theta \Lambda \Xi \Pi \Sigma \Upsilon \Phi \Chi \Psi \Omega
  \alpha \beta \gamma \delta \epsilon \zeta \eta \theta \iota \kappa \lambda \mu \nu \xi \pi \rho \sigma \tau \upsilon \phi \chi \psi \omega
  \varphi\varepsilon\varsigma
}\]
\[\mathup{
  \text{upright: }
  \alpha \beta \gamma \delta \epsilon \zeta \eta \theta \iota \kappa \lambda \mu \nu \xi \pi \rho \sigma \tau \upsilon \phi \chi \psi \omega
  \varphi\varepsilon\varsigma
}\]
\[\mathbf{\mathbfup{
  \text{bfup: }
  \Gamma \Delta \Theta \Lambda \Xi \Pi \Sigma \Upsilon \Phi \Chi \Psi \Omega
  \alpha \beta \gamma \delta \epsilon \zeta \eta \theta \iota \kappa \lambda \mu \nu \xi \pi \rho \sigma \tau \upsilon \phi \chi \psi \omega
  \varphi\varepsilon\varsigma
}}\]
\subsection{Accents for Math symbols}
\begin{align*}
\text{bar}:&\quad\let\Q\bar      \Q a\Q b\Q c\Q d\Q e\Q f\Q g\Q h\Q i\Q j\Q k\Q l\Q m\Q n\Q o\Q p\Q q\Q r\Q s\Q t\Q u\Q v\Q w\Q x\Q y\Q z\\
\text{overline}:&\quad\let\Q\overline \Q a\Q b\Q c\Q d\Q e\Q f\Q g\Q h\Q i\Q j\Q k\Q l\Q m\Q n\Q o\Q p\Q q\Q r\Q s\Q t\Q u\Q v\Q w\Q x\Q y\Q z\\
\text{hat}:&\quad\let\Q\hat      \Q a\Q b\Q c\Q d\Q e\Q f\Q g\Q h\Q i\Q j\Q k\Q l\Q m\Q n\Q o\Q p\Q q\Q r\Q s\Q t\Q u\Q v\Q w\Q x\Q y\Q z\\
\text{tilde}:&\quad\let\Q\tilde    \Q a\Q b\Q c\Q d\Q e\Q f\Q g\Q h\Q i\Q j\Q k\Q l\Q m\Q n\Q o\Q p\Q q\Q r\Q s\Q t\Q u\Q v\Q w\Q x\Q y\Q z\\
\text{wide-hat}:&\quad\let\Q\widehat  \Q a\Q b\Q c\Q d\Q e\Q f\Q g\Q h\Q i\Q j\Q k\Q l\Q m\Q n\Q o\Q p\Q q\Q r\Q s\Q t\Q u\Q v\Q w\Q x\Q y\Q z\\
\text{wide-tilde}:&\quad\let\Q\widetilde\Q a\Q b\Q c\Q d\Q e\Q f\Q g\Q h\Q i\Q j\Q k\Q l\Q m\Q n\Q o\Q p\Q q\Q r\Q s\Q t\Q u\Q v\Q w\Q x\Q y\Q z\\
\text{}:&\quad\widehat{abc}\widetilde{abc}
\end{align*}

\section{Big math}
\begin{equation}
 \sqrt{\sqrt{\sqrt{\sqrt{\sqrt{\sqrt{\sqrt{\sqrt{\frac ab}}}}}}}} \bar v=\iiint\overline{v}
\end{equation}
\begin{equation}
 \hat{\bar{x}} + \overline{\acute{y}} = ☉♁☾⛢♂
\end{equation}

\section{Verbatim}

\filepath|filename_test.pdf| has some mini code \code|my $x = $__self__|.

\section{Notes}
\lipsum[1]

\Note{hogehogei $h^{h^h}$ \lipsum[1]}

\Remark{hogehoge!}

\begin{DownPara}
\lipsum[2]

\lipsum[1]
\end{DownPara}

\lipsum[1]

Automatic note after this paragraph\addnote{Automatic note after this paragraph}.
Multiple notes\addnote{ more than two, as well} is possible.

\OutputNote

Automatic note after this paragraph\addnote{Automatic note after this paragraph}.
Multiple notes\addnote{more than two, as well} is possible.

\OutputNote
\OutputNote
\OutputNote
\OutputNote

\begin{align}
1+1=2\addnote*{even in eqs, but only once}
\end{align}
\OutputNote


\begin{DownPara}
\paragraph{Lipsum}\lipsum[2]

\lipsum[1]
\end{DownPara}

\paragraph{Lipsum}
\lipsum\lipsum\lipsum


\chapter{Minted}

\lipsum[1]

\texttt{This is a code with typewriter font.}
\begin{minted}{wolfram}
ClearProcess[];
topologies = CreateTopologies[1, 1->2, ExcludeTopologies->{Internal}];
diagrams = InsertFields[topologies, {F[2,{2}]}->{F[2,{2}], V[1]}, ExcludeParticles->{S, V[2|3|4|10]}];
oneQED = CalcFeynAmp[OffShell[CreateFeynAmp[diagrams],3->q]];
!$R_{\epsilon^{\prime\prime}}$! = RotationMatrix[!$\epsilon^{\prime\prime}$!, {0, 1, 0}];
!$R_{\theta}$!  = RotationMatrix[!$\theta$!, {0, 0, 1}];
!$R_{\delta}$!  = RotationMatrix[!$\delta$!, {1, 0, 0}];
\end{minted}
\begin{DownPara}
 The code should be aligned with \code|DownPara| env.
\end{DownPara}
\Note{And \code|Note| env.}



\begin{thebibliography}{9}
\bibitem{hogehoge}
Donald E. Knuth (1986) \emph{The \TeX{} Book}, Addison-Wesley Professional.
\end{thebibliography}

\end{document}

